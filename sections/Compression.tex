\section{Compression}

\begin{mainbox}
    {Definition - Code} 
    A code \(C\) is a mapping from outcomes to codewords
    \[C: \{1, ..., m\} \to \{0, 1\}^*\]
    \begin{itemize}[label=-]
        \item If there is no codeword that is a prefix of another codeword, the code is a \textbf{prefix code}. 
        \item Prefix codes retain injectivity when concatenating codewords.
    \end{itemize}
\end{mainbox}
Sets of codewords fulfilling the prefix property can be uniquely represented by the leaves of a binary tree. 
Since a leaf node has no children the prefix property is guaranteed.

\begin{mainbox}
    {Kraft's Inequality}
    If \(\{c_1, ...,c_m\}\) are codewords of a prefix code, then 
\begin{align}
    \sum_{x}2^{-l_x} \leq 1, \text{ where }l_x = |c_x|
\end{align}
Conversely, given \(\{l_1, ..., l_m\} \subset \N\) satisfying (3), there exists a prefix code with those codeword lengths.
\end{mainbox}
\begin{itemize}
    \item Codes for which Kraft's inequality is strict can be optimized by codeword pruning.
    \item A prefix is succinct, if Kraft's inequality holds with a equality.
    \item Succinct codes uniquely define a dyadic probabilistic model
    \[q(x) = 2^{- l_x}\]
    \item Expected codeword length of a prefix code \(C\) 
    \[L(C) = \sum_{x}p(x) l_x = \sum_{x}p(x) (-\log q(x)) = H(p; q)\]
    \item Using \(H(p; q) = H(p) + D(p || q)\) we can deduce that the minimal \(L(C)\) for a binary prefix code \(C\) is
    \[L* = H(p) + \min_{q: \text{dyadic}} D(p || q)\]
    \item Thus the closer \(q\) is to \(p\), the more optimal the prefix code is. 
    But since \(p\) doesn't have to be dyadic there can be an inherent suboptimality based on rounding.
\end{itemize}

\begin{mainbox}
    {Weak Law of Large Numbers}
    Let \(Y_1, ..., Y_n\) be iid. random variables with mean \(\mu\). Then 
    \[\overline{Y}_n := \frac{1}{n}\sum_{i = 1}^n Y_i \xrightarrow[]{\P} \mu \iff \lim_{n \to \infty}\P(|\overline{Y}_n - \mu| < \varepsilon) = 1, \forall \varepsilon > 0\]
\end{mainbox}

\textbf{Typicality - Asymptotic Equipartition}

Let \(X_1, ..., X_n \overset{\text{iid}}{\sim} p\). The \(\varepsilon\)-typical outcomes are 
\[\mathcal{A}_\varepsilon^n = \left\{x \in \{1, ..., m\}^n:\left|H(p)+ \frac{1}{n}\sum_{i = 1}^n \log p(x_i)\right| < \varepsilon\right\}\]

By the law of large numbers for any \(p, \varepsilon > 0\) and \(\delta > 0\), there exists an \(n_0\), s.t. \(\forall n \geq n_0\)
\[\P(A_\varepsilon^n)>1- \delta\]
in particular for \(\delta = \varepsilon\).

For all \(p, \varepsilon>0\) and \(n \in \N\), let \(x \in \mathcal{A}_\varepsilon^n\), then
\[2^{-n(H(p)+\varepsilon)} \leq p(x) \leq 2^{-n(H(p)-\varepsilon)}\]
\[(1-\varepsilon)2^{n(H(p)-\varepsilon)}\leq|\mathcal{A}_\varepsilon^n| \leq 2^{n(H(p)+\varepsilon)}\]
\[\implies |\mathcal{A}_\varepsilon^n| \approx 2^{nH(p)} \text{ and for }x \in \mathcal{A}_\varepsilon^n: p(x) \approx 2^{-nH(p)} \]

We define the AEP Code to encode whole sequences
\[\text{AEP}_\varepsilon^n = \begin{cases}
    0 B^n(x) &\text{if }x \notin \mathcal{A}_\varepsilon^n\\
    1 C^n(x) &\text{otherwise}
\end{cases}\]
where we enumerate over the typical and atypical sequences.

Then the average codeword length amortized over the encoding of the sequence \(x\) of \(n\) outcomes is
\[\frac{1}{n}|C_\varepsilon^n(x)| \leq \frac{1}{n}(1+\log|\mathcal{A}_\varepsilon^n|) \leq H(p)+ \frac{1}{n} + \varepsilon\]
\[\frac{1}{n}|B^n(x)| \leq \frac{1}{n}(1 + \log m^n) \leq \log m + \frac{1}{n}\]
This result is theoretically optimal but practically not very useful.

\textbf{Huffman Codes}

Let \(X\) have outcomes \(\{1, ..., m\}\) ordered (wlog) st. \(p(1) \geq ... \geq p(m)\). 

The Huffman contraction \(X'\) of \(X\) is defined as 
\[X' = \min\{m-1, X\}\]
We define the Huffman Code \(C\) for \(X\) recursively from \(C'\) for the H. contraction \(X'\)
\[C(x) = \begin{cases}
    x-1 &\text{if }m = 2\\
    C'(x)0 &\text{if }x = m-1 \land m > 2\\
    C'(x-1)1 &\text{if } x = m \land m > 2\\
    C'(x) &\text{otherwise} 
\end{cases}\]

Let \(C\) be a length-optimal code, then 
\[p(x) > p(x') \implies l_x \leq l_{x'}\]
\[\forall c \in \text{Img}(C) \text{ wt. } |c| \text{ maximal}: \exists c' \in \text{Img}(C). \ c' \text{ sibling of }c \]

Assume \(p_i\) ordered as above. Then a length-optimal prefix code \(C\) with 
\(l_1 \leq ...\leq l_{m-1} = l_m\) and \(c_{m-1}, c_m\) only differing in last bit, is called \textbf{canonical}.

Huffman codes are length-optimal.
