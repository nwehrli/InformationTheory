\section{Universal Coding}

A CDF \(F_X\) induces a partition of \([0; 1)\) into 
\[\{I_x: x \in X(\Omega)\}, \qquad I_x := [F_X(x) - p_X(x); F_X(x))\]
For \(I = [a, b) \subseteq I\) there exists \(z \in I, (z)_2 = 0.z_1z_2...z_l\)
with \(l = \lceil - \log(b-a)\rceil\) (\textit{there's short representation for every Interval \(I\)}).

\textbf{Shannon-Fano-Elias Codes}
\begin{itemize}[label=-]
    \item pick midpoint \(z_x = \sum_{x' < x} p_X(x') + \frac{1}{2}p_X(x)\)
    \item truncate to \(\lceil - \log p_X(x) \rceil + 1\) bits
\end{itemize}
Prefix-free, but not wasteful. Idea can be translated into 

\textbf{Arithmetic Coding}

Note that Huffman Codes require knowledge of the distribution and cannot easiy be adapted to changing distributions.

Consider a stochastic process made of \(X_t: \Omega \to A, A := \{0, ..., m-1\}\). We define 
\[Z = (0.X_1X_2...)_m \in [0, 1] \subset \R\]
For any \(Z\). If \(F_Z\) is a \textbf{bijection}, then \(U := F_Z(Z) \sim \mathcal{U}([0,1])\).

Then \(F_U(u) = u\) and \(U = (0.U_1U_2...) \implies U_i \overset{iid}{\sim} \text{Bernoulli}\left(\frac{1}{2}\right)\).

\begin{itemize}[label=-]
    \item encoder \(x_1, x_2,... \mapsto z = (0.x_1x_2...)_m \overset{F_Z}{\mapsto} u = (0.u_1u_2...)_2\)
    \item decoder \(u \overset{F_Z^{-1}}{\mapsto} z = (0.x_1x_2...)_m \mapsto x_1x_2\)
\end{itemize}

Consider \(X =(X_1, ...,X_n)\) with \(X: \Omega \to A^n\) and thus \(m^n\) possible outcomes.

\(F_X\) induces a partitioning \(\{I_x: x = (x_1, ..., x_n)\}\) as defined above.

The \textbf{arithmetic code} is given by the S.-F.-E. code for \(x\).

For the arithmetic code \(C\), we have
\[\lim_{n \to \infty} \frac{1}{n}\E(|C(X_1, ..., X_n)|) = H(\{X_t\})\]
\textbf{Incremental refinement} of \(I_x\) for \(x=(x_1,...,x_n)\) into \(m\) subinterval using 
\[\P(X_{n+1} | X_1 = x_1, ..., X_n = x_n)\] 
\textbf{Lempel-Ziv Code}

We denote a string to be compressed as 
\[x = x_{-s}x_{-s+1}...x_{-1}x_0x_1...x_t\]
where \(x_0, ..., x_t\) still needs to be encoded.

\textbf{Matching.}
\[\text{match}(x) := \{(j, l) \mid x_{-j}...x_{-j+l-1} = x_{0}...x_{l-1} \land j, l \geq 1\}\]
The \textbf{Maximal Matching} is \((j^*, l^*) \in \text{match}(x)\) is maximal in \(l\) and as tiebreaker minimal in \(j\).

\textbf{LZ77.}
\[c(x) = \begin{cases}
    (0, x_0)c(x_1...x_t) & \text{if match}(x) = \emptyset\\
    (1, j^*, l^*)c(x_{l^*}...x_t) & \text{otherwise}
\end{cases}\]
An integer \(x\) can be encoded with \(\leq \log x + 2 \log \log x + 4\) bits.
\[C'(x) = 00...01(x)_2, \quad |C'(x)| = 2 \lceil \log x \rceil + 1\]
and we construct
\[C(x) = C'(\lceil \log x\rceil)1(x)_2, \quad |C(x)| \leq \log x + 2 \log \log x + 4\]
LZ77 is optimal (i.e. reaches the shannon limit).
